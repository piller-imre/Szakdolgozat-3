% !TEX encoding = UTF-8 Unicode

\documentclass[a4paper,12pt]{article}

% Set margins
\usepackage[hmargin=3cm, vmargin=3cm]{geometry}

\frenchspacing

% Language packages
\usepackage[utf8]{inputenc}
\usepackage[T1]{fontenc}
\usepackage[magyar]{babel}

% AMS
\usepackage{amssymb,amsmath}

% Graphic packages
\usepackage{graphicx}

% Colors
\usepackage{color}
\usepackage[usenames,dvipsnames]{xcolor}

% Enumeration
\usepackage{enumitem}

% Links
\usepackage{hyperref}

\linespread{1.2}

\begin{document}

\pagestyle{empty}

\section*{Summary}

This work presents the topic of artistic filters with their mathematical foundation and some of my own filter implementations.

This is the first time when I work with image processing algorithms. I was always interested how they work. In the first couple of months I did several background research about algorithms and their mathematical background. I have shown the results of this research in Chapter 3.

I have designed and implemented 4 filters for cartoon, pencil and painting-like filtering. They have considered from theoretical and practical aspect of view. I have used the available literature, but they are based on my original ideas. After the backgroud researches, their mathematical formulation has given, and I have to find the appropriate software tools for the implementation.

I never used the OpenCV library before. The usage of the C++ programming language seems to be the proper solution. (Initially, I started to code in C, but some algorithms are implemented in C++, which were necessary for my filters.) The algorithms of the library are easy to use, as we can see in Chapter 5. We have to provide the appropriate parameters and we have reach the desired operation.

I have checked the calculation time of the filtering steps. It reveals the time consuming filtering operations. Some aspects of the video and real-time image processing requires further researches and development. The vibrating noise in the videos (as mentioned in Chapter 6 and can be checked by running the software) should be also filtered. I have proposed some solutions for reducing this type of noise, but their detailed consideration is out of the scope of this work.

\end{document}
