% !TEX encoding = UTF-8 Unicode

\Chapter{Tesztek, eredmények bemutatása}
\label{chap:tests}

Azokat a szűrőket amelyeket a \textbf{4. fejezet}ben tárgyaltam, nem csak képek átalakítására használtam, hanem videókon is teszteltem. Ebben a fejezetben szeretném leírni ezzel kapcsolatban a tapasztalataimat. 

Először a technikai adatait szeretném leírni a számítógépnek amin írtam és futtattam a programokat. Ez egy MacBook Pro  2014 mid, 2.6 GHz Intel Core i5 processzorral, 8 gb memóriával, Intel Iris 1536 MB grafikus kártyával és macOS High Sierra operációs rendszerrel.

\Section{Szűrők tesztelése képeken}

Elsőként képeken teszteltem a szűrőimet, azokon a képeken amiket már a \textbf{4. fejezet}ben bemutattam. Az első tesztekben azt mutatom meg, hogy mennyi milliszekundum alatt készül el a kép és melyik művelet mennyi ideig tart, természetesen milliszekundumban mutatom meg aztokat is.

\textbf{Cartoon-style filter}

Image loading time ms: 4.795\\
Creating windows time in ms: 48.967\\
Gaussian pyramid down in ms: 1.114\\
Bilateral filter time in ms: 22.23\\
Gaussian pyramid up time in ms: 1.387\\
Convert rgb img to gray and median filter  time in ms: 8.844\\
Adaptive threshold time in ms: 1.077\\
Convert back to color, bit-AND time in ms: 2.813\\
Write and show images time in ms: 21.723\\
Image processing time in ms: 113.229\\\\

\begin{figure}
\centering
\begin{tikzpicture}
\pie[text=legend, sum=113.229]{
4.795/image loading,
48.967/window creation,
1.114/Gaussian pyramid down,
22.23/bilateral filter,
1.387/Gaussian pyramid up,
8.844/RGB img to gray,
1.077/threshold,
2.813/convert to color,
21.723/write and show image
}
\end{tikzpicture}
\caption{Cartoon style filter}
\end{figure}

\textbf{Pencil sketch filter}

Image loading time in ms: 9.775\\
Creating a window time in ms: 51.585\\
Convert rgb iamge to gray and median filter time in ms: 1.495\\
Gaussian filter time in ms: 9.691\\
Gaussian filter and median filter divide time in ms: 0.476\\
Contrast strech time in ms: 0.168\\
Multiply the canvas and the smooth image time in ms: 2.793\\
Write and show images time in ms: 17.503\\
Image processing time in ms: 93.661

\textbf{Cartoon filter}

Image loading time in ms: 4.959\\
Create windows time in ms: 51.136\\
Median filtering time in ms: 9.916\\
Laplacian edge detectation time in ms: 0.961\\
Thresholding time in ms: 0.467\\
Copy mask to the image time in ms: 0.838\\
Write and show images time in ms: 19.56\\
Image processing time in ms: 88.059\\\\

\textbf{Aquarelle-style filter}

Image loading time in ms: 6.432\\
Creating windows time in ms: 50.489\\
Avarage blur time in ms: 1.26\\
Mean shift segmentation time in ms: 483.213\\
Write and show image time in ms: 13.407\\
Image processing time in ms: 554.927

\Section{Szűrők tesztelése videókon és valós időben}

Ebben a részben, a videók valamint az élő kép feldolgozási idejét mérem képkocka per másodpercben, azaz fps-ben. Illetve az esetleges képi hibákról és azok kijavításási lehetőségeiről ejtek néhány szót.

\textbf{Cartoon-style filter}

Ha futtatjuk az első filtert, látható, hogy a videó képe vibrál néhány pontban. Ezeket kilehet javítani, ha veszünk egy bizonyos kernel méretet és megvizsgáljuk, látjuk hogy a kernelen belül a színek túl nyomó többségben megegyeznek, viszont van néhány pont ami fekete, akkor a fekete pontokat átállítjuk olyan színüre, ami többségben van a kernelen belül. A videóknál, valamint a valós idejű feldolgozásnál mértem képkocka per másodpercet is. Látszik, hogy nem éri el a 24 fps-t sem a videó, sem a kamera képe ennél a szűrőnél. Azért 24 fps mivelt azt már az emberi szem folyamatosnak érzékeli, itt viszont kissé lassabb a kép. A méréseim alapján a videókon átlagosan 15-16 közötti képkocka szám van, a kamera képén 12-15.

\textbf{Pencil sketch filter}

Számításaim alapján ez a szűrő volt az, ami már a videókon majdnem elérte a kellő fps számot, hogy folyamatos legyen. Itt videókon 20-21 között volt a képkocka szám, valós időben pedig 15-16. Itt nem volt annyira sok az egyes műveletek számítási ideje.

\textbf{Cartoon filter}

Erről a szűrőről is elmondható ugyan az mint a Cartoon-style filterről, hogy a kép vibrál. Itt is alkalmazható lehetne az a megoldás, amit ott már leírtam. Itt a képkocka szám másodpercenként a videókban 17-18 között volt, valós időben 13-14 között.

\textbf{Aquarelle-style filter}

Mint látható a képernyőnkön, ha futtatjuk a programot videóval vagy a saját kameránk képével eléggé "szaggat", vagyis a képkocka szám per másodperc, nagyon alacsony. Láthatjuk, hogy az előző képi tesztekben is ennek a szűrőnek volt a legnagobb feldolgozási ideje. A mean shift szegmentáció olyan mértékű számítás igényt jelent a programnak, hogy nem tudja teljesíteni a kívánt fps számot a számítógépem. Itt átlagosan a videón 3,2-3,5 közötti képkocka számot mértem, élő képen 2,5-2,8.

%FPS counter - https://ariandy1.wordpress.com/2013/02/19/calculating-fps-in-opencv-for-live-capture/
%4-6 oldal
