% !TEX encoding = UTF-8 Unicode

\Chapter{Tesztek, eredmények bemutatása}
\label{chap:tests}

Azokat a szűrőket amelyeket \aref{chap:filters}. fejezetben tárgyaltam, nem csak képek átalakítására használtam, hanem videókon is teszteltem. Ebben a fejezetben szeretném leírni ezzel kapcsolatban a tapasztalataimat.  Fontosnak tartottam, hogy a vizsgálatok a futási időkre is kitérjenek, ezért a feldolgozási lépések számítási idejét külön-külön lemértem.

A tesztekhez használt számítógép konfigurációja a következő:
\begin{itemize}
\item MacBook Pro 2014 mid,
\item 2.6 GHz Intel Core i5 processzor,
\item 8GB RAM,
\item Intel Iris 1536 MB grafikus kártya,
\item macOS High Sierra operációs rendszer.
\end{itemize}

\Section{Szűrők tesztelése képeken}

Először képeken teszteltem az elkészített szűrőket. Az első tesztekben azt mutatom meg, hogy mennyi milliszekundum alatt készül el a kép és melyik művelet mennyi ideig tart, szintén milliszekundumban megadva.

A teszteknél a cél az, hogy megmutassa a korábban vizsgált szűrők számítási idejét. Ehhez az eddigi képek kerültek felhasználásra. A bemutatott példák számítási idejeit láthatjuk a következő szakaszokban.

\newpage

\SubSection{Cartoon-style filter}

A szűrő egyes lépéseinek időigényét \aref{fig:pie1}. ábrán láthatjuk. Az aktuális program esetében jól láthatóan időigényes művelet volt a kép betöltése, illetve a megjelenítéshez az ablak létrehozása.

A lényegi lépések közül a bilaterális szűr alkalmazása vette el a legtöbb időt, majd a szürkeárnyalatosra alakítás.

% Image loading time ms: 4.795\\
% Creating windows time in ms: 48.967\\
% Gaussian pyramid down in ms: 1.114\\
% Bilateral filter time in ms: 22.23\\
% Gaussian pyramid up time in ms: 1.387\\
% Convert rgb img to gray and median filter  time in ms: 8.844\\
% Adaptive threshold time in ms: 1.077\\
% Convert back to color, bit-AND time in ms: 2.813\\
% Write and show images time in ms: 21.723\\
% Image processing time in ms: 113.229\\\\

\begin{figure}[h!]
\centering
\begin{tikzpicture}
\pie[text=legend, sum=113.229]{
    4.795/képbetöltés (4.795 ms),
    48.967/ablak létrehozása (48.967 ms),
    1.114/Gauss-piramis down szűrő (1.114 ms),
    22.23/bilateral filter (22.23 ms),
    1.387/Gauss piramis up szűrő (1.387 ms),
    8.844/szürkeárnyalatosítás (8.844 ms),
    1.077/küszöbölés (1.077 ms),
    2.813/színesre konvertálás (2.813 ms),
    21.723/kép mentése és megjelenítése (21.723 ms)
}
\end{tikzpicture}
\caption{Cartoon style filter alkalmazásának időigénye (összesen 113.229 ms)}
\label{fig:pie1}
\end{figure}

\SubSection{Pencil sketch filter}

A szűrő szemléltetésére készített alkalmazás futásának jelentős idejét itt is a kép betöltése és az ablak megjelenítése tette ki (\ref{fig:pie2}. ábra). A következő nagyobb költségű művelet a Gauss szűrő alkalmazása volt.

% Image loading time in ms: 9.775\\
% Creating a window time in ms: 51.585\\
% Convert rgb iamge to gray and median filter time in ms: 1.495\\
% Gaussian filter time in ms: 9.691\\
% Gaussian filter and median filter divide time in ms: 0.476\\
% Contrast strech time in ms: 0.168\\
% Multiply the canvas and the smooth image time in ms: 2.793\\
% Write and show images time in ms: 17.503\\
% Image processing time in ms: 93.661

\begin{figure}[h!]
\centering
\begin{tikzpicture}
\pie[text=legend, sum=93.661]{
    9.775/képbetöltés (9.775 ms),
    51.585/ablak létrehozása (51.585 ms),
    1.495/szürkeárnyalatosítás és medián szűrő (1.495 ms),
    9.691/Gauss szűrő (9.691 ms),
    0.476/Gauss szűrt és medián szűrt kép osztása (0.476 ms),
    0.168/kontraszt széthúzása (0.168 ms),
    2.793/vászon és szűrt kép szorzása (2.793 ms),
    17.503/kép mentése és megjelenítése (17.503 ms)
}
\end{tikzpicture}
\caption{Pencil sketch filter alkalmazásának időigénye (összesen 93.661 ms)}
\label{fig:pie2}
\end{figure}

\SubSection{Cartoon filter}

A \textit{Cartoon filter} esetében láthatjuk, hogy a medián szűrés mennyire számításigényes művelet, mivel a kernelen belüli értékek rendezése szükséges hozzá (\ref{fig:pie3}. ábra). A Laplace élkiemelés, küszöbölés és a maszk alkalmazása is számítási időt tekintve lényegesen kisebb.

% Image loading time in ms: 4.959\\
% Create windows time in ms: 51.136\\
% Median filtering time in ms: 9.916\\
% Laplacian edge detectation time in ms: 0.961\\
% Thresholding time in ms: 0.467\\
% Copy mask to the image time in ms: 0.838\\
% Write and show images time in ms: 19.56\\
% Image processing time in ms: 88.059\\\\

\begin{figure}[h!]
\centering
\begin{tikzpicture}
\pie[text=legend, sum=88.059]{
    4.959/képbetöltés (4.959 ms),
    51.136/ablak létrehozása (51.136 ms),
    9.916/medián szűrő (9.916 ms),
    0.961/Laplace élkiemelés (0.961 ms),
    0.467/küszöbölés (0.467 ms),
    0.838/maszk alkalmazása (0.838 ms),
    19.56/kép mentése és megjelenítése (19.56 ms)
}
\end{tikzpicture}
\caption{Cartoon filter alkalmazásának időigénye (összesen 88.059 ms)}
\label{fig:pie3}
\end{figure}

\SubSection{Aquarelle-style filter}

A vízfestékszerű hatáshoz használt \textit{mean shift} szegmentálás nagyon számításigényes művelet, ahogy az a feldolgozási lépések számítási idejéből látszik (\ref{fig:pie4}. ábra). Az előzőekhez képest ez a szűrő az, amelyik a legtöbb számítást vette igénybe.

% Image loading time in ms: 6.432\\
% Creating windows time in ms: 50.489\\
% Avarage blur time in ms: 1.26\\
% Mean shift segmentation time in ms: 483.213\\
% Write and show image time in ms: 13.407\\
% Image processing time in ms: 554.927

\begin{figure}[h!]
\centering
\begin{tikzpicture}
\pie[text=legend, sum=554.927]{
    6.432/kép betöltése (6.432 ms),
    50.489/ablak létrehozása (50.489 ms),
    1.26/átlagoló szűrő (1.26 ms),
    483.213/\textit{mean shift} szegmentálás (483.213 ms),
    13.407/kép mentése és megjelenítése (13.407 ms)
}
\end{tikzpicture}
\caption{Aquarelle-style filter alkalmazásának időigénye (összesen 554.927 ms)}
\label{fig:pie4}
\end{figure}

\Section{Szűrők tesztelése videókon és valós időben}

Ebben a részben, a videók valamint az élő kép feldolgozási idejét mérem képkocka per másodpercben, azaz fps-ben. Illetve az esetleges képi hibákról és azok kijavításási lehetőségeiről ejtek néhány szót.

\SubSection{Cartoon-style filter}

Ha futtatjuk az első filtert, látható, hogy a videó képe vibrál néhány pontban.

Ezeket ki lehet javítani, ha veszünk egy bizonyos kernel méretet és megvizsgáljuk, látjuk hogy a kernelen belül a színek túl nyomó többségben megegyeznek, viszont van néhány pont ami fekete, akkor a fekete pontokat átállítjuk olyan színüre, ami többségben van a kernelen belül. A videóknál, valamint a valós idejű feldolgozásnál mértem képkocka per másodpercet is. Látszik, hogy nem éri el a 24 fps-t sem a videó, sem a kamera képe ennél a szűrőnél. Azért 24 fps mivelt azt már az emberi szem folyamatosnak érzékeli, itt viszont kissé lassabb a kép. A méréseim alapján a videókon átlagosan 15-16 közötti képkocka szám van, a kamera képén 12-15.

\SubSection{Pencil sketch filter}

Számításaim alapján ez a szűrő volt az, ami már a videókon majdnem elérte a kellő fps számot, hogy folyamatos legyen. Itt videókon 20-21 között volt a képkocka szám, valós időben pedig 15-16. Itt nem volt annyira sok az egyes műveletek számítási ideje.

\SubSection{Cartoon filter}

Erről a szűrőről is elmondható ugyan az mint a Cartoon-style filterről, hogy a kép vibrál. Itt is alkalmazható lehetne az a megoldás, amit ott már leírtam. Itt a képkocka szám másodpercenként a videókban 17-18 között volt, valós időben 13-14 között.

\SubSection{Aquarelle-style filter}

Mint látható a képernyőnkön, ha futtatjuk a programot videóval vagy a saját kameránk képével eléggé "szaggat", vagyis a képkocka szám per másodperc, nagyon alacsony. Láthatjuk, hogy az előző képi tesztekben is ennek a szűrőnek volt a legnagobb feldolgozási ideje. A mean shift szegmentáció olyan mértékű számítás igényt jelent a programnak, hogy nem tudja teljesíteni a kívánt fps számot a számítógépem. Itt átlagosan a videón 3,2-3,5 közötti képkocka számot mértem, élő képen 2,5-2,8.

%FPS counter - https://ariandy1.wordpress.com/2013/02/19/calculating-fps-in-opencv-for-live-capture/
%4-6 oldal

