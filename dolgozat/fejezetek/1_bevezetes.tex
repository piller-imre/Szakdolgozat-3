% !TEX encoding = UTF-8 Unicode

\Chapter{Bevezetés}

A fényképek és videók készítése napjainkban már teljesen természetessé, megszokottá vált. Ennek köszönhetően népszerűvé váltak azok az eljárások és alkalmazások, melyek segítségével az eredeti felvételből valamilyen stilizált, művészi jellegű képet készíthetünk. Összefoglaló néven ezeket művészi szűrőknek nevezzük.

A dolgozat célja, hogy bemutassa ezen szűrők matematikai hátterét, megvalósítási módját, illetve saját készítésű szűrőket is felvonultasson.

Fiatalok körében eléggé elterjedtek a kép feldolgozással kapcsolatos alkalmazások, gondoljunk csak bele, hogy a napijainkban használatos közösségi alkalmazások mindegyikében megtalálható ez az opció. Akár arc felismeréssel képeken, valós időben és mozgóképeken, vagy akár előre kidolgozott szűrők segítségével, amik elmossák a szineket, kiemelik az éleket vagy valamilyen módon átalakítják a képeket. Ezekre is ki fogok térni a dolgozat során.
% népszerű alkalmazások

Részletezni szeretném a különböző képfeldolgozási műveletek matematikai hátterét is. Láthatók lesznek, a matematikai leírásoknál példák is, valamint ábrák.
% matematikai háttér

Saját készítésű szűrőim rajzfilm, ceruza rajz szerű, valamint festmény jellegűek. Szűrőket képekkel lépésről lépésre szeretném bemutatni, továbbá a C vagy C++ implementációját is részletezni fogom.
% saját algoritmusok

Mint említettem szűrőimet C vagy C++ programozási nyelven szeretném megírni, OpenCV könyvtár segítségével. Az OpenCV az egy ingyenes, képfeldolgozási algoritmusak tartalmazó könyvtár. A dolgozatom során, fogom részletezni magát a könyvtárat is.
% OpenCV

A végén szeretnék néhány tesztet is készíteni, azon szűrőkre amiket én fogok elkészíteni. A tesztek a szűrők feldolgozási idejére, valamint az egyes lépések feldolgozási idejére vonatkozik képek esetén. Tervben van az is, hogy a szűrőket megpróbálom megvalósítani úgy, hogy ne csak képek feldolgozására legyen alkalmas, hanem videón, esetleg valós időben is működjön. Ezek tesztelésére, egy képkocka per másodperc tesztet szeretnék alkalmazni, amivel kiderül, hogy egy másodperc alatt hány képkockát tudunk előállítani az egyes szűrők esetén.
% tesztek

A következő fejezeteben a művészi jellegű szűrőket szeretném prezentálni.
