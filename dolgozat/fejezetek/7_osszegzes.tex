% !TEX encoding = UTF-8 Unicode

\Chapter{Összegzés}

A dolgozat a művészi szűrők témakörét mutatta be azok matematikai modelljén és néhány saját szűrő megvalósítás segítségével.

A dolgozat megírása előtt még nem foglalkoztam sem képfeldolgozással, sem azokkal az algoritmusokkal amelyeket itt említettem és használtam. Mindig is érdekelt, hogy ezek hogyan működnek. Az első pár hónapban komoly háttérkutatásokat végeztem, mind az algoritmusok és azok matematikai hátterével kapcsolatban.

Mint az előző fejezetekben olvasható, négy saját szűrőt raktam össze amelyek, rajzfilm, ceruzarajz-szerű, továbbá festmény jellegűek. Ezek mindegyikének bemutattam a matematikáját, valamint az implementációját. Néhányat ezek közül, leírások segítségével állítottam össze, de volt olyan amelyek teljesen saját ötlet alapján készült. A filterek algoritmusainak matematikai háttere az előző háttérkutatás után már nem volt ismeretlen, így már csak egy eszköz kellett, amivel meg is lehetett valósítani ezeket. 

Korábban nem használtam az OpenCV könyvtárat. A C++ programozási nyelv használata tünt a legcélszerűbbnek. (Eleinte C-ben kezdtem a kódok írását, de több olyan algoritmus implementációja hiányzik az OpenCV könyvtárból ami C++-ban viszont megtalálható. Ezen algoritmusokat viszont fontosak voltak a saját készítésű filterekhez.) Az algoritmusok a könyvtárban könnyedén használhatók, mint ahogy a \aref{chap:implement}. fejezetben részletezem. Egyszerűen megadjuk a kívánt paramétereket és már is elértük a kívánt műveletet.

Tesztekkel meg sikerült vizsgálni, hogy a szűrők használata során az egyes lépések számítási ideje milyen. Ezáltal jól láthatóvá váltak a képfeldolgozás szempontjából költségesebb műveletek. A videókon, továbbá a valós időben való képfeldolgozás a saját készítésű szűrők esetén néhány helyen még további kutatást és fejlesztést igényelne. A videók képe vibrál, ahogy \aref{chap:tests}. fejezetben is említettem, de akár látható is a dolgozathoz csatolt CD-mellékleten, ha futtatjuk ezen alkalmazásokat. Ezen hibák javítására szerepelnek javaslatok a dolgozatban, viszont javításukhoz egy külön, teljes körű, célzottan erre a problémakörre koncentráló kutatásra lenne szükség.
